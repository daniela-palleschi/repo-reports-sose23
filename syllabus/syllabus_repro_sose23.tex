% Options for packages loaded elsewhere
\PassOptionsToPackage{unicode}{hyperref}
\PassOptionsToPackage{hyphens}{url}
\PassOptionsToPackage{dvipsnames,svgnames,x11names}{xcolor}
%
\documentclass[
  letterpaper,
  DIV=11,
  numbers=noendperiod]{scrartcl}

\usepackage{amsmath,amssymb}
\usepackage{lmodern}
\usepackage{iftex}
\ifPDFTeX
  \usepackage[T1]{fontenc}
  \usepackage[utf8]{inputenc}
  \usepackage{textcomp} % provide euro and other symbols
\else % if luatex or xetex
  \usepackage{unicode-math}
  \defaultfontfeatures{Scale=MatchLowercase}
  \defaultfontfeatures[\rmfamily]{Ligatures=TeX,Scale=1}
\fi
% Use upquote if available, for straight quotes in verbatim environments
\IfFileExists{upquote.sty}{\usepackage{upquote}}{}
\IfFileExists{microtype.sty}{% use microtype if available
  \usepackage[]{microtype}
  \UseMicrotypeSet[protrusion]{basicmath} % disable protrusion for tt fonts
}{}
\makeatletter
\@ifundefined{KOMAClassName}{% if non-KOMA class
  \IfFileExists{parskip.sty}{%
    \usepackage{parskip}
  }{% else
    \setlength{\parindent}{0pt}
    \setlength{\parskip}{6pt plus 2pt minus 1pt}}
}{% if KOMA class
  \KOMAoptions{parskip=half}}
\makeatother
\usepackage{xcolor}
\setlength{\emergencystretch}{3em} % prevent overfull lines
\setcounter{secnumdepth}{-\maxdimen} % remove section numbering
% Make \paragraph and \subparagraph free-standing
\ifx\paragraph\undefined\else
  \let\oldparagraph\paragraph
  \renewcommand{\paragraph}[1]{\oldparagraph{#1}\mbox{}}
\fi
\ifx\subparagraph\undefined\else
  \let\oldsubparagraph\subparagraph
  \renewcommand{\subparagraph}[1]{\oldsubparagraph{#1}\mbox{}}
\fi


\providecommand{\tightlist}{%
  \setlength{\itemsep}{0pt}\setlength{\parskip}{0pt}}\usepackage{longtable,booktabs,array}
\usepackage{calc} % for calculating minipage widths
% Correct order of tables after \paragraph or \subparagraph
\usepackage{etoolbox}
\makeatletter
\patchcmd\longtable{\par}{\if@noskipsec\mbox{}\fi\par}{}{}
\makeatother
% Allow footnotes in longtable head/foot
\IfFileExists{footnotehyper.sty}{\usepackage{footnotehyper}}{\usepackage{footnote}}
\makesavenoteenv{longtable}
\usepackage{graphicx}
\makeatletter
\def\maxwidth{\ifdim\Gin@nat@width>\linewidth\linewidth\else\Gin@nat@width\fi}
\def\maxheight{\ifdim\Gin@nat@height>\textheight\textheight\else\Gin@nat@height\fi}
\makeatother
% Scale images if necessary, so that they will not overflow the page
% margins by default, and it is still possible to overwrite the defaults
% using explicit options in \includegraphics[width, height, ...]{}
\setkeys{Gin}{width=\maxwidth,height=\maxheight,keepaspectratio}
% Set default figure placement to htbp
\makeatletter
\def\fps@figure{htbp}
\makeatother
\newlength{\cslhangindent}
\setlength{\cslhangindent}{1.5em}
\newlength{\csllabelwidth}
\setlength{\csllabelwidth}{3em}
\newlength{\cslentryspacingunit} % times entry-spacing
\setlength{\cslentryspacingunit}{\parskip}
\newenvironment{CSLReferences}[2] % #1 hanging-ident, #2 entry spacing
 {% don't indent paragraphs
  \setlength{\parindent}{0pt}
  % turn on hanging indent if param 1 is 1
  \ifodd #1
  \let\oldpar\par
  \def\par{\hangindent=\cslhangindent\oldpar}
  \fi
  % set entry spacing
  \setlength{\parskip}{#2\cslentryspacingunit}
 }%
 {}
\usepackage{calc}
\newcommand{\CSLBlock}[1]{#1\hfill\break}
\newcommand{\CSLLeftMargin}[1]{\parbox[t]{\csllabelwidth}{#1}}
\newcommand{\CSLRightInline}[1]{\parbox[t]{\linewidth - \csllabelwidth}{#1}\break}
\newcommand{\CSLIndent}[1]{\hspace{\cslhangindent}#1}

\KOMAoption{captions}{tableheading}
\makeatletter
\makeatother
\makeatletter
\makeatother
\makeatletter
\@ifpackageloaded{caption}{}{\usepackage{caption}}
\AtBeginDocument{%
\ifdefined\contentsname
  \renewcommand*\contentsname{Table of contents}
\else
  \newcommand\contentsname{Table of contents}
\fi
\ifdefined\listfigurename
  \renewcommand*\listfigurename{List of Figures}
\else
  \newcommand\listfigurename{List of Figures}
\fi
\ifdefined\listtablename
  \renewcommand*\listtablename{List of Tables}
\else
  \newcommand\listtablename{List of Tables}
\fi
\ifdefined\figurename
  \renewcommand*\figurename{Figure}
\else
  \newcommand\figurename{Figure}
\fi
\ifdefined\tablename
  \renewcommand*\tablename{Table}
\else
  \newcommand\tablename{Table}
\fi
}
\@ifpackageloaded{float}{}{\usepackage{float}}
\floatstyle{ruled}
\@ifundefined{c@chapter}{\newfloat{codelisting}{h}{lop}}{\newfloat{codelisting}{h}{lop}[chapter]}
\floatname{codelisting}{Listing}
\newcommand*\listoflistings{\listof{codelisting}{List of Listings}}
\makeatother
\makeatletter
\@ifpackageloaded{caption}{}{\usepackage{caption}}
\@ifpackageloaded{subcaption}{}{\usepackage{subcaption}}
\makeatother
\makeatletter
\@ifpackageloaded{tcolorbox}{}{\usepackage[many]{tcolorbox}}
\makeatother
\makeatletter
\@ifundefined{shadecolor}{\definecolor{shadecolor}{rgb}{.97, .97, .97}}
\makeatother
\makeatletter
\makeatother
\ifLuaTeX
  \usepackage{selnolig}  % disable illegal ligatures
\fi
\IfFileExists{bookmark.sty}{\usepackage{bookmark}}{\usepackage{hyperref}}
\IfFileExists{xurl.sty}{\usepackage{xurl}}{} % add URL line breaks if available
\urlstyle{same} % disable monospaced font for URLs
\hypersetup{
  pdftitle={Reproducible analysis reports with eye-tracking reading time data},
  colorlinks=true,
  linkcolor={blue},
  filecolor={Maroon},
  citecolor={Blue},
  urlcolor={Blue},
  pdfcreator={LaTeX via pandoc}}

\title{Reproducible analysis reports with eye-tracking reading time
data}
\author{}
\date{}

\begin{document}
\maketitle
\ifdefined\Shaded\renewenvironment{Shaded}{\begin{tcolorbox}[interior hidden, frame hidden, sharp corners, borderline west={3pt}{0pt}{shadecolor}, breakable, boxrule=0pt, enhanced]}{\end{tcolorbox}}\fi

\hypertarget{course-description-quarto}{%
\section{Course description Quarto}\label{course-description-quarto}}

This course will provide students with the skills and know-how needed to
create reproducible reports and presentations of eye-tracking reading
data. A brief introduction will be given into common measures in
eye-tracking reading and the importance of developing a reproducible
workflow, followed by hands-on exercises in RStudio with the R
programming language. The main skills developed include data wrangling
(with the tidyverse package), data visualisation (with the ggplot2
package), and running and communicating descriptive and inferential
statistics. By the end of the course, students will be able to apply
what they learned to a variety of data types in both academic and
professional settings. This course is aimed at students who have some
practical experience with R and RStudio, although this is not a strict
requirement. Students who cannot bring their own laptop to class should
contact the instructor as early as possible, so an alternative laptop
can be organised. The language of instruction is English.

At the end of the course, you will:

\begin{itemize}
\tightlist
\item
  have an understanding of common eye-tracking reading measures
\item
  be able to visualise reading times
\item
  be able to run linear mixed models on reading times
\item
  be able to set-up and maintain a reproducible workflow in RStudio
\end{itemize}

\hypertarget{schedule}{%
\section{Schedule}\label{schedule}}

\begin{longtable}[]{@{}
  >{\raggedright\arraybackslash}p{(\columnwidth - 4\tabcolsep) * \real{0.1282}}
  >{\raggedright\arraybackslash}p{(\columnwidth - 4\tabcolsep) * \real{0.3077}}
  >{\raggedright\arraybackslash}p{(\columnwidth - 4\tabcolsep) * \real{0.5513}}@{}}
\toprule()
\begin{minipage}[b]{\linewidth}\raggedright
Session
\end{minipage} & \begin{minipage}[b]{\linewidth}\raggedright
Date
\end{minipage} & \begin{minipage}[b]{\linewidth}\raggedright
Topic(s)
\end{minipage} \\
\midrule()
\endhead
1 & Wed. April 12, 2023

10:15 - 11:45 & Eye-tracking during reading

Reading: Vasishth, Malsburg, and Engelmann (2013) \\
2 & Wed. April 12, 2023

12:15 - 1:45 & Reproducible workflow with RStudio

Reading: \\
3 & Wed. April 12, 2023

2:15 - 3:45 & Working with a dataset

Reading: Biondo, Soilemezidi, and Mancini (2022) \\
4 & Thurs. April 13, 2023 & Data Wrangling and summary statistics \\
5 & Thurs. April 13, 2023 & Data Visualisation \\
6 & Thurs. April 13, 2023 & Communicating your results \\
7 & Fri. April 14, 2023 & (Generalised) linear mixed models \\
8 & Fri. April 14, 2023 & (Generalised) linear mixed models \\
9 & Fri. April 14, 2023 & Independent work / Q\&A \\
\multicolumn{3}{@{}>{\raggedright\arraybackslash}p{(\columnwidth - 4\tabcolsep) * \real{0.9872} + 4\tabcolsep}@{}}{%
} \\
10 & Fri. June 30, 2023 & \\
11 & Fri. June 30, 2023 & \\
12 & Sat. July 1, 2023 & \\
13 & Sat. July 1, 2023 & \\
14 & Sat. July 1, 2023 & \\
\bottomrule()
\end{longtable}

\hypertarget{refs}{}
\begin{CSLReferences}{1}{0}
\leavevmode\vadjust pre{\hypertarget{ref-biondo2022}{}}%
Biondo, Nicoletta, Marielena Soilemezidi, and Simona Mancini. 2022.
{``Yesterday Is History, Tomorrow Is a Mystery: An Eye-Tracking
Investigation of the Processing of Past and Future Time Reference During
Sentence Reading.''} \emph{Journal of Experimental Psychology: Learning,
Memory, and Cognition} 48 (7): 1001--18.
\url{https://doi.org/10.1037/xlm0001053}.

\leavevmode\vadjust pre{\hypertarget{ref-vasishth2013}{}}%
Vasishth, Shravan, Titus von der Malsburg, and Felix Engelmann. 2013.
{``What Eye Movements Can Tell Us about Sentence Comprehension.''}
\emph{Wiley Interdisciplinary Reviews: Cognitive Science} 4 (2):
125--34. \url{https://doi.org/10.1002/wcs.1209}.

\end{CSLReferences}



\end{document}
